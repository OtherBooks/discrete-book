\documentclass[12pt]{article}

\usepackage{discrete}

\def\thetitle{Sequences: Conclusion} % will be put in the center header on the first page only.
\def\lefthead{Math 228 Notes} % will be put in the left header
\def\righthead{\thetitle} % will be put in the right header




\begin{document}

\section{Chapter Summary}\label{sec:sequences-conc}

\begin{activity}
Each day your supply of magic chocolate covered espresso beans doubles (each one splits in half), but then you eat 5 of them.  You have 10 at the start of day 0.

\begin{questions}
\question Write out the first few terms of the sequence.  Then give a recursive definition for the sequence and explain how you know it is correct.
\question Prove, using induction, that the last digit of the number of beans you have on the $n$th day is always a 5 for all $n \ge 1$.
\question Find a closed formula for the $n$th term of the sequence and prove it is correct by induction.
\end{questions}

\end{activity}

In this chapter we explored sequences and mathematical induction.  At first these might not seem entirely related, but there is a link: recursive reasoning.  When we have many cases (maybe infinitely many), it is often easier to describe a particular case by saying how it relates to other cases, instead of describing it absolutely.  For sequences, we can describe the $n$th term in the sequence by saying how it is related to the \emph{previous} term.  When showing a statement involving the variable $n$ is true for all values of $n$, we can describe why the case for $n = k$ is true on the basis of why the case for $n = k-1$ is true.

While thinking of problems recursively is usually easer than thinking of them absolutely (at least after you get used to thinking in this way), our ultimate goal is to move beyond this recursive description.  For sequences, we want to find \emph{closed formulas} for the $n$th term of the sequence.  For proofs, we want to know the statement is actually true for a particular $n$ (not only under the assumption that the statement is true for the previous value of $n$). In this chapter we saw some methods for moving from recursive descriptions to absolute descriptions.

\begin{itemize}
\item If the terms of a sequence increase by a constant difference or constant ratio (these are both recursive descriptions), then the sequences are arithmetic or geometric, respectively, and we have closed formulas for each of these based on the initial terms and common difference or ratio.
\item If the terms of a sequence increase at a polynomial rate (that is, if the differences between terms form a sequence with a polynomial closed formula), then the sequence is itself given by a polynomial closed formula (of degree one more than the sequence of differences).
\item If the terms of a sequence increase at an exponential rate, then we expect the closed formula for the sequence to be exponential.  These sequences often have relatively nice recursive formulas, and the \emph{characteristic root technique} allows us to find the closed formula for these sequences.
\item If we want to prove that a statement is true for all values of $n$ (greater than some first small value), and we can describe why the statement being true for $n = k$ implies the statement is true for $n = k+1$, then the \emph{principle of mathematical induction} gives us that the statement is true for all values of $n$ (greater than the base case).
\end{itemize}

Throughout the chapter we tried to understand \emph{why} these facts listed above are true.  In part, that is what proofs, by induction or not, attempt to accomplish: they explain why mathematical truths are in fact truths.  As we develop our ability to reason about mathematics, it is a good idea to make sure that the methods of our reasoning are sound.  The branch of mathematics that deals with deciding whether reasoning is good or not is \emph{mathematical logic}, the subject of the next chapter.

\end{document}
